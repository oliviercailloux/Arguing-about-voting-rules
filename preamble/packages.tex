%INSTALL

%avoids a warning
%\usepackage[log-declarations=false]{xparse}
\usepackage{xparse}
%Should be inputed before xunicode, someone says. And put everything font-related before fontspec. But I don’t need it anyway.
% \usepackage{amssymb} %for e.g. succcurlyeq.
\usepackage{fontspec} %font selecting commands
%Call amssymb before amsmath or any other amslatex style files (amsthm, ...).
\usepackage{xunicode}
%warn about missing characters
\tracinglostchars=2

%REDAC
\usepackage[hang,flushmargin]{footmisc}%The 'hang' option flushes the footnote marker to the left margin of the page, while the 'flushmargin' option flushes the text as well.
\usepackage{booktabs}
\usepackage{hyphenat}
\usepackage{calc}

\usepackage{mathtools} %load this before babel!
	\mathtoolsset{showonlyrefs,showmanualtags}

\usepackage[super]{nth}
\usepackage{listings} %typeset source code listings
	\lstset{language=XML,tabsize=2,literate={"}{{\tt"}}1,captionpos=b}
\usepackage[nolist,printonlyused]{acronym}%when using smaller, we get: relsize Warning: Failed to get list of defined font sizes.
\usepackage{xspace}
\usepackage[textsize=small]{todonotes}
\usepackage[pdfpagelabels=false]{hyperref}
%pdfusetitle does not work!
%breaklinks makes links on multiple lines into different PDF links to the same target.
%colorlinks (false): Colors the text of links and anchors. The colors chosen depend on the the type of link. In spite of colored boxes, the colored text remains when printing.
%linkcolor=black: this leaves other links in colors, e.g. refs in green, don't print well.
%pdfborder (0 0 1, set to 0 0 0 if colorlinks): width of PDF link border
%hidelinks
\hypersetup{breaklinks,bookmarksopen}
% hyperref doc says: Package bookmark replaces hyperref’s bookmark organization by a new algorithm (...) Therefore I recommend using this package.
\usepackage{bookmark}
\usepackage[capitalise,noabbrev]{cleveref}

% center floats by default, but do not use with float
% \usepackage{floatrow}
% \makeatletter
% \g@addto@macro\@floatboxreset\centering
% \makeatother
\usepackage{enumitem} %follow enumerate by a string saying how to display enumeration
\usepackage{ragged2e} %new com­mands \Cen­ter­ing, \RaggedLeft, and \RaggedRight and new en­vi­ron­ments Cen­ter, FlushLeft, and FlushRight, which set ragged text and are eas­ily con­fig­urable to al­low hy­phen­ation (the cor­re­spond­ing com­mands in LaTeX, all of whose names are lower-case, pre­vent hy­phen­ation al­to­gether). 
\usepackage{siunitx} %[expproduct=tighttimes, decimalsymbol=comma]
\usepackage{braket} %for \Set
\usepackage{natbib}
\usepackage{doi}

\usepackage{amsmath,amsthm}
% \usepackage{amsfonts} %not required?
% \usepackage{dsfont} %for what?
%unicode-math overwrites the following commands from the mathtools package: \dblcolon, \coloneqq, \Coloneqq, \eqqcolon. Using the other colon-like commands from mathtools will lead to inconsistencies. Plus, Using \overbracket and \underbracke from mathtools package. Use \Uoverbracket and \Uunderbracke for original unicode-math definition.
%use exclusively \mathbf and choose math bold style below.
\usepackage[warnings-off={mathtools-colon, mathtools-overbracket}, bold-style=ISO]{unicode-math}

%IJCAI demands Adobe’s Times Roman. The example paper actually uses Nimbus Roman No 9 L (very close); with CM for the sans font. I use texgyretermes, a modern version of NR9L and texgyrecursor for the monospaced version. No instruction provided for math font. I use the default (latin modern).
%\setsansfont[%
%	BoldFont = texgyreheros-bold.otf,%
%	ItalicFont = texgyreheros-italic.otf,%
%	BoldItalicFont = texgyreheros-bolditalic.otf,%
%	Mapping=tex-text% to turn "--" into dashes, useful for bibtex%%
%]{texgyreheros-regular.otf}

\newfontfamily\texgyretermesfamily[
	BoldFont = texgyretermes-bold.otf,%
	ItalicFont = texgyretermes-italic.otf,%
	BoldItalicFont = texgyretermes-bolditalic.otf,%
	Mapping=tex-text% to turn "--" into dashes, useful for bibtex%%
]{texgyretermes-regular.otf}

\newfontfamily\xitsfamily[
	BoldFont = xits-bold.otf,%
	ItalicFont = xits-italic.otf,%
	BoldItalicFont = xits-bolditalic.otf,%
	Mapping=tex-text% to turn "--" into dashes, useful for bibtex%%
]{xits-regular.otf}

%\setmonofont[%
%       Fractions=On,
%       BoldFont = texgyrecursor-bold.otf,%
%       ItalicFont = texgyrecursor-italic.otf,%
%       BoldItalicFont = texgyrecursor-bolditalic.otf,%
%       Mapping=tex-text% to turn "--" into dashes, useful for bibtex%%
%]{texgyrecursor-regular.otf}

%\setmainfont[%
%       Fractions=On,
%       BoldFont = texgyretermes-bold.otf,%
%       ItalicFont = texgyretermes-italic.otf,%
%       BoldItalicFont = texgyretermes-bolditalic.otf,%
%       Mapping=tex-text% to turn "--" into dashes, useful for bibtex%%
%]{texgyretermes-regular.otf}

%NOT loading the font (which should result in latinmodern-math being automatically loaded) solves the spacing problem with some subscripts.
%\setmathfont{latinmodern-math.otf}

% setminus is missing in lmmath font and in texgyrepagella-math
\AtBeginDocument{\renewcommand{\setminus}{\smallsetminus}}
%\setmathfont[range={"29F5}]{xits-math.otf}
%∉
%\setmathfont[range={"2209}]{texgyrepagella-math.otf}
%mapsto
%\setmathfont[range={"21A6}]{texgyrepagella-math.otf}
%✗
%\setmathfont[range={"2717}]{texgyrepagella-regular.otf}
%✓
%\setmathfont[range={"2713}]{texgyrepagella-math.otf}
%\vert, symbol code found in unimath-symbols.pdf (also used for \lvert and \rvert)
%\setmathfont[range={"007C}]{texgyrepagella-math.otf}

%GRAPHICS
\usepackage{pgf}
\usepackage{pgfplots}
\pgfplotsset{compat=1.14}
	\usetikzlibrary{matrix,fit,plotmarks,calc,trees,shapes.geometric,positioning,plothandlers}
\usepackage{graphicx}

\graphicspath{{graphics/},{graphics-dm/}}
\DeclareGraphicsExtensions{.pdf}

%HACKING
\usepackage{printlen}
\uselengthunit{mm}
% 	\newlength{\templ}% or LenTemp?
% 	\setlength{\templ}{6 pt}
% 	\printlength{\templ}
\usepackage{etoolbox} %for addtocmd command
\usepackage{scrhack}% load at end. Corrects a bug in float package, which is outdated but might be used by other packages
\usepackage{xltxtra} %somebody said that this is loaded by fontspec, but does not seem correct: if not loaded explicitly, does not appear in the log and \showhyphens is not corrected.

%Beamer-specific
%\usepackage{todonotes}%UNNECESSARY
%\usepackage{natbib}%REMOVE?
%ADD
% \usepackage{appendixnumberbeamer}
% \newcommand{\citep}{\cite}
% \setbeamersize{text margin left=0.1cm, text margin right=0.1cm} 
% \setbeamertemplate{navigation symbols}{} 
% to solve: this theme produces lots of warnings!
% \usetheme{BrusselsBelgium}
% \usefonttheme{professionalfonts}

