\RequirePackage[l2tabu, orthodox]{nag}
\RequirePackage{silence}
\WarningFilter{ifpdf}{Someone has redefined \pdfoutput}
\WarningFilter{fmtcount}{\ordinal already defined use \FCordinal instead}
\documentclass[english]{beamer}
%INSTALL

%avoids a warning
%\usepackage[log-declarations=false]{xparse}
\usepackage{xparse}
%Should be inputed before xunicode, someone says. And put everything font-related before fontspec. But I don’t need it anyway.
% \usepackage{amssymb} %for e.g. succcurlyeq.
\usepackage{fontspec} %font selecting commands
%Call amssymb before amsmath or any other amslatex style files (amsthm, ...).
\usepackage{xunicode}
%warn about missing characters
\tracinglostchars=2

%REDAC
\usepackage[hang,flushmargin]{footmisc}%The 'hang' option flushes the footnote marker to the left margin of the page, while the 'flushmargin' option flushes the text as well.
\usepackage{booktabs}
\usepackage{hyphenat}
\usepackage{calc}

\usepackage{mathtools} %load this before babel!
	\mathtoolsset{showonlyrefs,showmanualtags}

\usepackage[super]{nth}
\usepackage{listings} %typeset source code listings
	\lstset{language=XML,tabsize=2,literate={"}{{\tt"}}1,captionpos=b}
\usepackage[nolist,printonlyused]{acronym}%when using smaller, we get: relsize Warning: Failed to get list of defined font sizes.
\usepackage{xspace}
\usepackage[textsize=small]{todonotes}
\usepackage[pdfpagelabels=false]{hyperref}
%pdfusetitle does not work!
%breaklinks makes links on multiple lines into different PDF links to the same target.
%colorlinks (false): Colors the text of links and anchors. The colors chosen depend on the the type of link. In spite of colored boxes, the colored text remains when printing.
%linkcolor=black: this leaves other links in colors, e.g. refs in green, don't print well.
%pdfborder (0 0 1, set to 0 0 0 if colorlinks): width of PDF link border
%hidelinks
\hypersetup{breaklinks,bookmarksopen}
% hyperref doc says: Package bookmark replaces hyperref’s bookmark organization by a new algorithm (...) Therefore I recommend using this package.
\usepackage{bookmark}
\usepackage[capitalise,noabbrev]{cleveref}

% center floats by default, but do not use with float
% \usepackage{floatrow}
% \makeatletter
% \g@addto@macro\@floatboxreset\centering
% \makeatother
\usepackage{enumitem} %follow enumerate by a string saying how to display enumeration
\usepackage{ragged2e} %new com­mands \Cen­ter­ing, \RaggedLeft, and \RaggedRight and new en­vi­ron­ments Cen­ter, FlushLeft, and FlushRight, which set ragged text and are eas­ily con­fig­urable to al­low hy­phen­ation (the cor­re­spond­ing com­mands in LaTeX, all of whose names are lower-case, pre­vent hy­phen­ation al­to­gether). 
\usepackage{siunitx} %[expproduct=tighttimes, decimalsymbol=comma]
\usepackage{braket} %for \Set
\usepackage{natbib}
\usepackage{doi}

\usepackage{amsmath,amsthm}
% \usepackage{amsfonts} %not required?
% \usepackage{dsfont} %for what?
%unicode-math overwrites the following commands from the mathtools package: \dblcolon, \coloneqq, \Coloneqq, \eqqcolon. Using the other colon-like commands from mathtools will lead to inconsistencies. Plus, Using \overbracket and \underbracke from mathtools package. Use \Uoverbracket and \Uunderbracke for original unicode-math definition.
%use exclusively \mathbf and choose math bold style below.
\usepackage[warnings-off={mathtools-colon, mathtools-overbracket}, bold-style=ISO]{unicode-math}

%IJCAI demands Adobe’s Times Roman. The example paper actually uses Nimbus Roman No 9 L (very close); with CM for the sans font. I use texgyretermes, a modern version of NR9L and texgyrecursor for the monospaced version. No instruction provided for math font. I use the default (latin modern).
%\setsansfont[%
%	BoldFont = texgyreheros-bold.otf,%
%	ItalicFont = texgyreheros-italic.otf,%
%	BoldItalicFont = texgyreheros-bolditalic.otf,%
%	Mapping=tex-text% to turn "--" into dashes, useful for bibtex%%
%]{texgyreheros-regular.otf}

\newfontfamily\texgyretermesfamily[
	BoldFont = texgyretermes-bold.otf,%
	ItalicFont = texgyretermes-italic.otf,%
	BoldItalicFont = texgyretermes-bolditalic.otf,%
	Mapping=tex-text% to turn "--" into dashes, useful for bibtex%%
]{texgyretermes-regular.otf}

\newfontfamily\xitsfamily[
	BoldFont = xits-bold.otf,%
	ItalicFont = xits-italic.otf,%
	BoldItalicFont = xits-bolditalic.otf,%
	Mapping=tex-text% to turn "--" into dashes, useful for bibtex%%
]{xits-regular.otf}

%\setmonofont[%
%       Fractions=On,
%       BoldFont = texgyrecursor-bold.otf,%
%       ItalicFont = texgyrecursor-italic.otf,%
%       BoldItalicFont = texgyrecursor-bolditalic.otf,%
%       Mapping=tex-text% to turn "--" into dashes, useful for bibtex%%
%]{texgyrecursor-regular.otf}

%\setmainfont[%
%       Fractions=On,
%       BoldFont = texgyretermes-bold.otf,%
%       ItalicFont = texgyretermes-italic.otf,%
%       BoldItalicFont = texgyretermes-bolditalic.otf,%
%       Mapping=tex-text% to turn "--" into dashes, useful for bibtex%%
%]{texgyretermes-regular.otf}

%NOT loading the font (which should result in latinmodern-math being automatically loaded) solves the spacing problem with some subscripts.
%\setmathfont{latinmodern-math.otf}

% setminus is missing in lmmath font and in texgyrepagella-math
\AtBeginDocument{\renewcommand{\setminus}{\smallsetminus}}
%\setmathfont[range={"29F5}]{xits-math.otf}
%∉
%\setmathfont[range={"2209}]{texgyrepagella-math.otf}
%mapsto
%\setmathfont[range={"21A6}]{texgyrepagella-math.otf}
%✗
%\setmathfont[range={"2717}]{texgyrepagella-regular.otf}
%✓
%\setmathfont[range={"2713}]{texgyrepagella-math.otf}
%\vert, symbol code found in unimath-symbols.pdf (also used for \lvert and \rvert)
%\setmathfont[range={"007C}]{texgyrepagella-math.otf}

%GRAPHICS
\usepackage{pgf}
\usepackage{pgfplots}
\pgfplotsset{compat=1.14}
	\usetikzlibrary{matrix,fit,plotmarks,calc,trees,shapes.geometric,positioning,plothandlers}
\usepackage{graphicx}

\graphicspath{{graphics/},{graphics-dm/}}
\DeclareGraphicsExtensions{.pdf}

%HACKING
\usepackage{printlen}
\uselengthunit{mm}
% 	\newlength{\templ}% or LenTemp?
% 	\setlength{\templ}{6 pt}
% 	\printlength{\templ}
\usepackage{etoolbox} %for addtocmd command
\usepackage{scrhack}% load at end. Corrects a bug in float package, which is outdated but might be used by other packages
\usepackage{xltxtra} %somebody said that this is loaded by fontspec, but does not seem correct: if not loaded explicitly, does not appear in the log and \showhyphens is not corrected.

%Beamer-specific
%\usepackage{todonotes}%UNNECESSARY
%\usepackage{natbib}%REMOVE?
%ADD
% \usepackage{appendixnumberbeamer}
% \newcommand{\citep}{\cite}
% \setbeamersize{text margin left=0.1cm, text margin right=0.1cm} 
% \setbeamertemplate{navigation symbols}{} 
% to solve: this theme produces lots of warnings!
% \usetheme{BrusselsBelgium}
% \usefonttheme{professionalfonts}


\newcommand{\R}{ℝ}
\newcommand{\N}{ℕ}
\newcommand{\Z}{ℤ}
\newcommand{\card}[1]{\lvert{#1}\rvert}
\newcommand{\powerset}[1]{\mathcal{P}(#1)}
\newcommand{\suchthat}{\;\ifnum\currentgrouptype=16 \middle\fi|\;}

\AtBeginDocument{%
	\renewcommand{\epsilon}{\varepsilon}
}

% with amssymb, but I don’t want to use amssymb just for that.
% \newcommand{\restr}[2]{{#1}_{\restriction #2}}
%\newcommand{\restr}[2]{{#1\upharpoonright}_{#2}}
\newcommand{\restr}[2]{{#1|}_{#2}}%sometimes typed out incorrectly within \set.
%\newcommand{\restr}[2]{{#1}_{\vert #2}}%\vert errors when used within \Set and is typed out incorrectly within \set.
\DeclareMathOperator*{\argmax}{arg\,max}


%Voting
\newcommand{\allalts}{\mathcal{A}}
%\newcommand{\allaltsexcept}{\allalts^-}
\newcommand{\allaltsexcept}{\allalts \setminus \set{z}}
\newcommand{\alts}{A}
%the alts on which a specific profile is defined
%\newcommand{\thesealts}[1][\prof]{\alts_{#1}}
\newcommand{\thesealts}{\allalts}
\newcommand{\allvoters}{\mathcal{N}}
\newcommand{\allsystems}{\mathcal{G}}
\newcommand{\prof}{\mathbf{R}}
\newcommand{\tprof}{\prof^*}
\newcommand{\allprofs}{\mathbfcal{R}}
\newcommand{\zerov}{\mathbf{0}}
%all Hamiltonian cycles that need to be considered
\newcommand{\allH}{\mathcal{S}}

\newcommand{\pbasic}[1]{\prof^{#1}_\epsilon}
\newcommand{\pelem}[1]{\prof^{#1}_e}
\newcommand{\pcycl}[1]{\prof^{#1}_c}
\newcommand{\pcycllong}[1]{\prof^{#1}_{cl}}
\newcommand{\pinv}[1][\prof]{\overline{#1}}
\newcommand{\dmap}{{\xitsfamily δ}}
%powerset without zero
\newcommand{\powersetz}[1]{\mathcal{P}_\emptyset(#1)}

%logic atom
\DeclareDocumentCommand{\lato}{ O{\prof} O{\alts} }{[#1 \!\mapsto\! #2]} % UE: added 2 x \! to make it tighter
%logic atom in
\newcommand{\tightoverset}[2]{%
  \mathop{#2}\limits^{\vbox to -.5ex{\kern-0.9ex\hbox{$#1$}\vss}}}
\DeclareDocumentCommand{\latoin}{ O{\prof} O{\alpha} }{[#1 \tightoverset{\in}{⟼} #2]}
\newcommand{\alllang}{\mathcal{L}}
\newcommand{\ltru}{\texttt{T}}
\newcommand{\lfal}{\texttt{F}}
\newcommand{\laxiom}[1]{{\texgyretermesfamily\textsc{#1}}}
\newcommand{\betatr}{\hat{\beta}}
\newcommand{\betasp}{\boxed{β}}
\newcommand{\deltasp}{\boxed{δ}}
\newcommand{\deltacsp}{\boxed{δ\text{-cycl}}}
\newcommand{\kernel}[1]{\mathcal{K}(#1)}
%\DeclareMathOperator{\increas}{inc}
\newcommand{\increas}{\ordaltsrestr{\allaltsexcept}}
\newcommand{\ordaltsrestr}[1]{\restr{{\ordalts}}{#1}}
\newcommand{\ordalts}{\succ}
\newcommand{\ordaltsrev}{\prec}
\newcommand{\ordaltsreveq}{\preceq}
\newcommand{\smlex}{<_{\text{lex}}}

%ARG TH
\newcommand{\AF}{\mathcal{AF}}
\newcommand{\labelling}{L}
\newcommand{\labin}{\textbf{in}\xspace}
\newcommand{\labout}{\textbf{out}}
\newcommand{\labund}{\textbf{undec}\xspace}
\newcommand{\nonemptyor}[2]{\ifthenelse{\equal{#1}{}}{#2}{#1}}
\newcommand{\gextlab}[2][]{
	\labelling{\mathcal{GE}}_{(#2, \nonemptyor{#1}{\ibeatsr{#2}})}
}

\newcommand{\jext}{\hat{j}}

%MISC
\newcommand{\ibeats}{\vartriangleright}
\newcommand{\lequiv}{\Vvdash}
\newcommand{\weightst}{W^{\,t}}
\newcommand{\inferrule}[2]{
	\ensuremath{
		\begin{array}{c}
			#1\\
			\hline
			#2\\
		\end{array}
	}
}
\newcommand{\inferruled}[3]{
	\ensuremath{
		\begin{array}{c|c}
			\multicolumn{2}{c}{#1}\\
			\hline
			#2 & #3\\
		\end{array}
	}
}

%Sorting
\newcommand{\cats}{\mathcal{C}}
\newcommand{\catssubsets}{2^\cats}
\newcommand{\catgg}{\vartriangleright}
\newcommand{\catll}{\vartriangleleft}
\newcommand{\catleq}{\trianglelefteq}
\newcommand{\catgeq}{\trianglerighteq}
\newcommand{\alttoc}[2][x]{(#1 \xrightarrow{} #2)}
\newcommand{\alttocat}[3]{(#2 \xrightarrow{#1} #3)}
\newcommand{\alttoI}{(x \xrightarrow{} \left[\underline{C_x}, \overline{C_x}\right])}
\newcommand{\alttocatdm}[3][t]{\left(#2 \thinspace \raisebox{-3pt}{$\xrightarrow{#1}$}\thinspace #3\right)}
\newcommand{\alttocatatleast}[2]{\left(#1 \thinspace \raisebox{-3pt}{$\xrightarrow[]{≥}$}\thinspace #2\right)}
\newcommand{\alttocatatmost}[2]{\left(#1 \thinspace \raisebox{-3pt}{$\xrightarrow[]{≤}$}\thinspace #2\right)}

\input{preamble/redac}
\input{preamble/draw}
\input{preamble/acronyms}

\title{Arguing about voting rules}
\subject{Voting rules}
\keywords{social choice}
\author[Olivier Cailloux]{\emph{Olivier Cailloux} \inst{1} \and Ulle Endriss \inst{2}}
\institute[LAMSADE]{\inst{1} LAMSADE, Paris Dauphine \and \inst{2} ILLC, University of Amsterdam}
\date{\formatdate{14}{11}{2017}}

\begin{document}
\begin{frame}[plain]
	\tikz[remember picture,overlay]{
		\node[anchor=south west, inner sep=0] at (current page.south west) {
			{\includegraphics[height=1cm]{LAMSADE95.jpg}}
		};
		\path (current page.south) node[anchor=south, inner sep=0] {
			{\includegraphics[height=1cm]{Dauphine.jpg}}
		};
		\node[anchor=south east, inner sep=0] at (current page.south east) {
			{\includegraphics[height=1cm]{illc_cropped}}
		};
		\path (current page.south) ++ (0, 4em) node[anchor=south, inner sep=0] {
			\scriptsize\url{https://github.com/oliviercailloux/Arguing-about-voting-rules}
		};
	}
   \titlepage
\end{frame}
\addtocounter{framenumber}{-1}

\begin{frame}
	\frametitle{Outline}
	\tableofcontents[hideallsubsections, sectionstyle=shaded/show]
\end{frame}

\section{Context}
\subsection{Introduction}
\begin{frame}
	\frametitle{Introduction}
	
	\begin{block}{Context}
	\begin{itemize}
		\item Voting rule: a systematic way of aggregating different opinions and decide
		\item Multiple reasonable ways of doing this
		\item Different voting rules have different interesting properties
		\item None satisfy all desirable properties
		\item Some are more reasonable than others
	\end{itemize}
	\end{block}
	\begin{block}{Our goal}
		We want to easily communicate about strength and weaknesses of voting rules.
	\end{block}
\end{frame}

\begin{frame}[fragile]
	\frametitle{Voting rule}
	
	\begin{description}[Profile (on $A$)]
		\item[Alternatives] $\allalts = \set{a, b, c, d, \ldots}$%; $\card{\allalts}=m$
		\item[Possible voters] $\allvoters = \set{1, 2, \ldots}$
		\item[Voters] $\emptyset \subset \voters \subseteq \allvoters$
%		\item[Linear orders on $A \subseteq \allalts$] $\linors$.
		\item[Profile] function $\prof$ from $\voters$ to linear orders on $\allalts$.
		\item[Voting rule] function $f$ mapping each $\prof$ to winners $\emptyset \subset A \subseteq \allalts$.
	\end{description}
	\vfill
	\begin{center}
		\begin{tikzpicture}
			\path node[profile matrix] (profile) {
				\prof(1)&
				\prof(2)
				\\
				| (profile11) | a&
				b
				\\
				b&
				a
				\\
				c&
				| (profile32) | c
				\\
			};
			\path ($(profile.south west)!.5!(profile.south east)$) ++ (0, -5mm) node {$\prof$};
			
			\path node[draw, rectangle, fit=(profile11) (profile32), outer xsep=2mm, outer ysep=1mm] (justprofile) {};
			\path (justprofile.east) ++ (2.5cm, 0) node[inner sep=0] (winners) {\mbox{} $A = \Set{a, b}$};
			\path[draw, ->] (justprofile.east) to[bend left=35] node[anchor=south] {$f$} (winners.west);

			\path[draw, decorate, decoration={brace, mirror}] (justprofile.south west) -- (justprofile.south east);
		\end{tikzpicture}
	\end{center}
\end{frame}

\subsection{Two voting rules}
\begin{frame}
	\frametitle{Borda}
Given a profile $\prof$:
	\begin{itemize}
		\item count the score of each alternative;
		\item the highest scores win.
		\item Score of $a \in \allalts$ is the number of alternatives it beats.
	\end{itemize}
	
	\begin{equation}
		\prof =
		\begin{array}{ccc}
			a	&	b	&	b\\
			d	&	c	&	a\\
			c	&	a	&	c\\
			b	&	d	&	d
		\end{array}.
	\end{equation}
	\begin{minipage}{7cm}
		\begin{itemize}
			\item score $a$ is\dots? \pause $3 + 1 + 2 = 6$
			\item score $b$ is $0 + 3 + 3 = 6$
			\item score $c$ is $1 + 2 + 1 = 4$
			\item score $d$ is $2 + 0 + 0 = 2$
		\end{itemize}
	\end{minipage}%
	\begin{minipage}{4cm}
		Winners are $\Set{a, b}$.
	\end{minipage}
\end{frame}

\begin{frame}
	\frametitle{Condorcet’s principle}
	\begin{block}{Condorcet’s principle}
		We ought to take the Condorcet winner as sole winner if it exists.
		\begin{itemize}
			\item $a$ \emph{beats} $b$ iff more than half the voters prefer $a$ to $b$.
			\item $a$ is a \emph{Condorcet winner} iff $a$ beats every other alternatives.
		\end{itemize}
	\end{block}
	\vfill
	\begin{equation}
		\prof =
		\begin{array}{ccc}
			a	&	b	&	b\\
			d	&	c	&	a\\
			c	&	a	&	c\\
			b	&	d	&	d
		\end{array}.\text{ Who wins?}\pause {b}
	\end{equation}
\end{frame}

\subsection{Our objective}
\begin{frame}
	\frametitle{How are voting rules analyzed?}
	
	\begin{itemize}
		\item Examples featuring counter-intuitive results for some voting rules.
		\item Properties of voting rules, \eg Borda does not satisfy Condorcet’s principle.
		\item Axiomatization of a voting rule: accepting such principles lead to a unique voting rule.
	\end{itemize}
\end{frame}

\begin{frame}
	\frametitle{Our objective}
	
	\begin{itemize}
		\item Different voting rules
		\item Arguments in favor or against rules
		\item Dispersed in the literature
		\item Using mathematical formalism
	\end{itemize}
	\begin{block}{We propose}
		\begin{itemize}
			\item Common language
			\item Instantiate arguments on concrete examples
		\end{itemize}
	\end{block}
	Goal: help understand strengths and weaknesses of given rules.
\end{frame}

\AtBeginSection{
	\begin{frame}
		\frametitle{Outline}
		\tableofcontents[currentsection, hideallsubsections]
	\end{frame}
}

\section{Two examples}
\subsection{Natural language example}
\begin{frame}
	\frametitle{Example}
	
	\begin{block}{Who should win?}
		\begin{center}
			$\begin{array}{lc}
				\text{Voter 1:} & a\succ b\succ c \\
				\text{Voter 2:} & a\succ b\succ c \\
				\text{Voter 3:} & c\succ b\succ a 
			\end{array}$
		\end{center}
	\end{block}
	\begin{itemize}
		\item Veto rule chooses $b$
		\item Borda rule chooses $a$
	\end{itemize}
\end{frame}

\begin{frame}
	\begin{center}
		\fbox{%
			$\begin{array}{lc}
				\text{\textcolor{red}{Voter 1:}} & \textcolor{red}{a\succ b\succ c} \\
				\textcolor{darkgreen}{\text{Voter 2:}} & \textcolor{darkgreen}{ a\succ b\succ c} \\
				\textcolor{darkgreen}{\text{Voter 3:}} & \textcolor{darkgreen}{ c\succ b\succ a} 
			\end{array}$%
		}
	\end{center}
	
	\begin{tabular}{@{}lp{6.8cm}r@{}}
		\textbf{System:} & Take the \textcolor{red}{\textit{red subprofile}}. Here, \textcolor{red}{\textit{$a$ should win}}, right? & \textcolor{gray}{[unanimity]} \\
		\textbf{User:} & Obviously! & \\
		\textbf{System:} & Now consider the \emph{green subprofile}. For symmetry reasons, there should be a \emph{three-way tie}, right? & \textcolor{gray}{[cancellation]} \\
		\textbf{User:} & Sounds reasonable. & \\
		\textbf{System:} &  So, as there was a three-way tie for the green part, the red part should decide the overall winner, right? & \textcolor{gray}{[reinforcement]} \\
		\textbf{User:} & Yes. \\
		\textbf{System:} & To summarise, you agree that $a$ should win. 
	\end{tabular}
\end{frame}

\subsection{Logical example}
\begin{frame}
	\frametitle{Language}
	
	We use propositional logic (with connectives $¬, ∨, ∧, →$).
	\begin{block}{Atoms}
		\begin{itemize}
			\item One atom for each $(\prof, A)$, $\emptyset \subset A \subseteq \allalts$.
			\item An atom talks about assigning winners $A$ to $\prof$.
			\item Written $\lato$.
		\end{itemize}
	\end{block}
	\begin{block}{Semantics}
		Semantics $v_f$, given a voting rule $f$:
		\begin{equation}
			v_f(\lato) = \ltru \text{ iff } f(\prof)=A.
		\end{equation}
	\end{block}
\end{frame}

\begin{frame}
	\frametitle{Dominance l-axiom}
	
	\begin{definition}[\laxiom{Dom}]
		L-axiom \laxiom{Dom}: for each $\prof$,
		\setlength\abovedisplayskip{1 ex}
		\begin{equation}
			\latoin[\prof][\powersetz{U_\prof}],
		\end{equation}
		with $U_\prof$ the set of alternatives in $\prof$ that are not dominated.
	\end{definition}
\end{frame}

\begin{frame}
	\frametitle{Symmetric cancellation l-axiom}
	
	\begin{definition}[\laxiom{Sym}]
		For each $\prof$ consisting of a linear order and its inverse,
		\setlength\abovedisplayskip{1 ex}
		\begin{equation}
			\lato[\prof][\allalts].
		\end{equation}
	\end{definition}
	\begin{example}
		$\prof_1 =
		\begin{array}{cc}
			a&c\\
			b&b\\
			c&a\\
		\end{array},$
	constraints? $f(\prof_1) = \pause \allalts = \set{a, b, c}.$ \pause
	\end{example}
	\begin{example}
		$\prof_2 =
		\begin{array}{cc}
			a&b\\
			b&a\\
			c&c\\
		\end{array}$,
		constraints? \pause None.
	\end{example}
\end{frame}

\begin{frame}
	\frametitle{Reinforcement axiom}
	
	Classical reinforcement axiom: consider $\prof_1$, $\prof_2$,
	\begin{itemize}
		\item having winners $A_1$, $A_2$,
		\item with $A_1 \cap A_2 ≠ \emptyset$;
	\end{itemize}
	then winners in $\prof_1 + \prof_2$ must be $A_1 \cap A_2$.
	\begin{example}
		\begin{equation}
			\prof_1 =
			\begin{array}{cc}
				a&b\\
				b&a\\
				c&c\\
			\end{array},
			A_1 = \set{a, b},
			\pause
			\prof_2 =
			\begin{array}{ccc}
				a&b&a\\
				b&a&c\\
				c&c&b\\
			\end{array},
			A_2 = \set{a},
		\end{equation}
		\begin{equation}
			\prof =
			\begin{array}{ccccc}
				a&b&a&b&a\\
				b&a&b&a&c\\
				c&c&c&c&b\\
			\end{array}.
			\text{ Winners? }
			\pause
			\set{a}
		\end{equation}
	\end{example}
\end{frame}

\begin{frame}
	\frametitle{Reinforcement l-axiom}
	
	Classical reinforcement axiom: consider $\prof_1$, $\prof_2$,
	\begin{itemize}
		\item having winners $A_1$, $A_2$,
		\item with $A_1 \cap A_2 ≠ \emptyset$;
	\end{itemize}
	then winners in $\prof_1 + \prof_2$ must be $A_1 \cap A_2$.
	\begin{definition}[\laxiom{Reinf}]
		For each $\prof_1, \prof_2, A_1, A_2 \subseteq \allalts, A_1 \cap A_2 ≠ \emptyset$:
		\begin{equation}
			(\lato[\prof_1][A_1] ∧ \lato[\prof_2][A_2]) → \lato[\prof_1 + \prof_2][A_1 \cap A_2].
		\end{equation}
	\end{definition}
\end{frame}

\begin{frame}
	\frametitle{A simple argument}
	
	\begin{block}{Claim}
		\begin{minipage}{2cm}
			Consider:
		\end{minipage}
		\begin{minipage}{\columnwidth-3cm}
			\begin{itemize}
				\item
					$
					\prof =
					\begin{array}{cccc}
						a&b&a&c\\
						b&c&b&b\\
						c&a&c&a\\
					\end{array}$,
				\item $J = \Set{\laxiom{Dom}, \laxiom{Sym}, \laxiom{Reinf}}$.
			\end{itemize}
		\end{minipage}
		\vspace{\baselineskip}

		We can prove that for $f$ compliant with $J$:
		\setlength\abovedisplayskip{1 ex}
		\begin{equation}
			\latoin[\prof][\set{\set{a}, \set{b}, \set{a, b}}].
		\end{equation}
	\end{block}
	See how?
	\pause
	Consider
	$
	\prof_D =
	\begin{array}{cc}
		a&b\\
		b&c\\
		c&a\\
	\end{array}, 
	\prof_S =
	\begin{array}{cc}
		a&c\\
		b&b\\
		c&a\\
	\end{array}, 
	\prof = \prof_D + \prof_S
	$.
\end{frame}

\begin{frame}
	\frametitle{Example proof}
	$
	\prof_D =
	\begin{array}{cc}
		a&b\\
		b&c\\
		c&a\\
	\end{array}, 
	\prof_S =
	\begin{array}{cc}
		a&c\\
		b&b\\
		c&a\\
	\end{array}, 
	\prof = \prof_D + \prof_S = 
					\begin{array}{cccc}
						a&b&a&c\\
						b&c&b&b\\
						c&a&c&a\\
					\end{array}
	$.
	\begin{enumerate}
		\item $\latoin[\prof_D][\set{\set{a}, \set{b}, \set{a, b}}]$ (\laxiom{dom})
			\item $\lato[\prof_S][\set{a, b, c}]$ (\laxiom{Sym})
			\item $((1) ∧ (2)) → \latoin[\prof][\set{\set{a}, \set{b}, \set{a, b}}]$ (\laxiom{Reinf-sets})
			\item $\latoin[\prof][\set{\set{a}, \set{b}, \set{a, b}}]$
	\end{enumerate}
\end{frame}

\section{Arguing for Borda}
\begin{frame}[fragile]
	\frametitle{Argument building for Borda}
	
	Write $f_B$ for the Borda rule.
	\begin{itemize}
		\item We want to produce an argument justifying Borda’s output.
		\item Given $\prof$, we want an argument with claim $\lato[\prof][f_B(\prof)]$.
		\item Basis: \citet{young_axiomatization_1974}’s axiomatization of the Borda rule.
		\item Our l-axiomatization uses three simple profile types plus \laxiom{reinf}.
	\end{itemize}
\end{frame}

\subsection{L-axiomatization}
\begin{frame}[fragile]
	\frametitle{Elementary profile}
	
	Fix an arbitrary linear order $k$ on $\allalts$. Given $A \subseteq \allalts$, define $\pelem{A}$.
	\begin{minipage}{(\columnwidth - 2em) * \real{0.58}}
		\begin{definition}[Elementary profile]
			\begin{tikzpicture}
				\path node[profile matrix, row 2/.style={nodes={draw, rectangle, minimum width=1.5em, minimum height=9ex}}, row 3/.style={nodes={draw, rectangle, minimum width=1.5em, minimum height=6ex}}] (profile) {
					\pelem{A}(1)	&\pelem{A}(2)\\
					| (profile-1-1) | 	& | (profile-1-2) | 	\\
					| (profile-2-1) | 	& | (profile-2-2) | 	\\
				};
				\path (profile-1-1.west) ++(-5mm, 0) node[anchor=east] (profile-1-1 expl) {$\restr{k}{A}$};
				\path (profile-1-2.east) ++(5mm, 0) node[anchor=west] (profile-1-2 expl) {$\restr{k^{-1}}{A}$};
				\path (profile-2-1.west) ++(-5mm, 0) node[anchor=east] (profile-2-1 expl) {$\restr{k}{\allalts \setminus A}$};
				\path (profile-2-2.east) ++(5mm, 0) node[anchor=west] (profile-2-2 expl) {$\restr{k^{-1}}{\allalts \setminus A}$};
				\path[draw, ->] (profile-1-1 expl.east) -- (profile-1-1.west);
				\path[draw, <-] (profile-1-2.east) -- (profile-1-2 expl.west);
				\path[draw, ->] (profile-2-1 expl.east) -- (profile-2-1.west);
				\path[draw, <-] (profile-2-2.east) -- (profile-2-2 expl.west);
				
				\path[draw, decorate, decoration={brace, mirror}] (profile.south west) -- (profile.south east);
				\path (profile.south) ++(0, -5mm) node {$\pelem{A}$};
			\end{tikzpicture}
		\end{definition}
	\end{minipage}\hspace{2em}%
	\begin{minipage}{(\columnwidth - 2em) * \real{0.42}}
		\begin{example}
			\begin{tikzpicture}
				\path node[profile matrix] (profile) {
					\pelem{\Set{a, b, c}}(1)	&\pelem{\Set{a, b, c}}(2)\\
					| (profile-1-1) | a	& | (profile-1-2) | c	\\
					b	&b	\\
					| (profile-3-1) | c	& | (profile-3-2) | a	\\[2ex]
					| (profile-4-1) | d	& | (profile-4-2) | e	\\
					| (profile-5-1) | e	& | (profile-5-2) | d	\\
				};
				\path node[draw, rectangle, fit=(profile-1-1) (profile-3-1)] {};
				\path node[draw, rectangle, fit=(profile-4-1) (profile-5-1)] {};
				\path node[draw, rectangle, fit=(profile-1-2) (profile-3-2)] {};
				\path node[draw, rectangle, fit=(profile-4-2) (profile-5-2)] {};
				
				\path (profile.south west) ++ (1mm, -1mm) coordinate (profile-bottom-left);
				\path (profile.south east) ++ (-1mm, -1mm) coordinate (profile-bottom-right);
				\path[draw, decorate, decoration={brace, mirror}] (profile-bottom-left) -- (profile-bottom-right);
				\path (profile.south) ++(0, -5mm) node {$\pelem{\Set{a, b, c}}$};
			\end{tikzpicture}
		\end{example}
	\end{minipage}
\end{frame}

\begin{frame}
	\frametitle{Cyclic profiles}
	Given $S$ a complete cycle in $\allalts$, define $\pcycl{S}$.
	\begin{definition}[Cyclic profile]
		$\pcycl{S}$ is the profile composed by all $\card{\allalts}$ possible linearizations of $S$ as preference orderings. 
	\end{definition}
	\begin{example}
		\begin{equation}
			\pcycl{\left<a, b, c, d\right>} =
			\begin{array}{cccc}
				a&b&c&d\\
				b&c&d&a\\
				c&d&a&b\\
				d&a&b&c\\
			\end{array}.
		\end{equation}
	\end{example}
\end{frame}

\begin{frame}
	\frametitle{Borda l-axiomatization}
\begin{description}
	\item[\laxiom{elem}] for all $A$: $\lato[\pelem{A}][A]$.
	\item[\laxiom{cycl}] for all $S$: $\lato[\pcycl{S}][\allalts]$.
	\item[\laxiom{reinf}] as previously but generalized to any number of summed profiles.
%	\item[reinf-sub] when $\prof_3 = \prof_1 - \prof_2$, and $f(\prof_2) = \alts_\prof$, $f(\prof_3)$ must equal $f(\prof_1)$. This is also a consequence of the classical reinforcement axiom.
	\item[\laxiom{canc}] cancellation: when all pairs of alternatives $(a, b)$ in a profile are such that $a$ is preferred to $b$ as many times as $b$ to $a$, the set of winners must be $\allalts$.
%	\item[simp] simplification: when a profile consists in a repetition of the same sub-profile, the sub-profile must have the same winners.
\end{description}
\end{frame}

\subsection{Example}
\begin{frame}
	\frametitle{An example}
	
	Consider $\allalts=\set{a, b, c, d}$ and a profile $\prof$ defined as:
	\begin{equation}
		\prof =
		\begin{array}{cc}
			a	&	c\\
			b	&	b\\
			d	&	a\\
			c	&	d
		\end{array}.
	\end{equation}
	\pause
	We want to justify that $f_B(\prof)=\Set{a, b}$.
\end{frame}

\begin{frame}
	\frametitle{Sketch}
	
	\begin{itemize}
		\item Consider any $\prof' = q_1 \pelem{a, b} + q_2 \pelem{a, b, c} + \sum_{S \in \mathcal{S}} q_S \pcycl{S}$, $q_1, q_2, q_S \in \N, \mathcal{S}$ some set of cycles.
		\item In $\prof'$, $W=\set{a, b}$ must win.
		\item Find $k \in \N$ such that $\overline{k \prof} + \prof'$ cancel.
		\item Then $k \prof$ has winners $W$. (Skipping details.)
		\item Then $\prof$ has winners $W$.
	\end{itemize}
	Our task: find $\prof'$ a combination of elementary and cyclic profiles such that $\overline{k \prof} + \prof'$ cancel.
	
	Good news: this is always possible.
\end{frame}

\begin{frame}
	\frametitle{Application on the example}
	
	Define $\prof' = \pelem{a, b} + 2 \pelem{a, b, c} + \pcycl{\left<c, b, a, d\right>} + \pcycl{\left<b, d, c, a\right>}$.
	\begin{enumerate}
		\item $\lato[\pelem{a, b}][\set{a, b}]$ (\laxiom{elem})
		\item $\lato[\pelem{a, b, c}][\set{a, b, c}]$ (\laxiom{elem})
		\item $\lato[\pcycl{\left<c, b, a, d\right>}][\allalts]$ (\laxiom{cycl})
		\item $\lato[\pcycl{\left<b, d, c, a\right>}][\allalts]$ (\laxiom{cycl})
		\item $\lato[\prof'][\set{a, b}]$ (\laxiom{reinf}, 1, 2, 3, 4)
		\item $\lato[4 \prof + \overline{4 \prof}][\allalts]$ (\laxiom{canc})
		\item $\lato[4 \prof + \overline{4 \prof} + \prof'][\set{a, b}]$ (\laxiom{reinf}, 5, 6)
		\item $\lato[\overline{4 \prof} + \prof'][\allalts]$ (\laxiom{canc})
		\item $\lato[4 \prof][\set{a, b}]$ (\laxiom{reinf}, 7, 8)
		\item $\lato[\prof][\set{a, b}]$ (\laxiom{reinf}, 9)
	\end{enumerate}
\end{frame}

\section[General goal]{Goal: Build argumentative and adaptative recommender systems}
\subsection{Disagreeing about Borda}
\begin{frame}
	\frametitle{Counter-argument against Borda}
	
	\begin{block}{Counter-argument against Borda}
		Not Condorcet-consistent!
	\end{block}
	\begin{example}
		\begin{equation}
			\prof =
			\begin{array}{ccc}
				a	&	b	&	b\\
				d	&	c	&	a\\
				c	&	a	&	c\\
				b	&	d	&	d
			\end{array}.
		\end{equation}
	\end{example}
	\begin{itemize}
		\item Argument against Borda: use a \laxiom{Cond} l-axiom
		\item Counter-argue with \laxiom{FvsC}.
	\end{itemize}
\end{frame}

\subsection{Argumentative systems}
\begin{frame}
	\frametitle{Building argumentative recommender systems}
	
	\begin{block}{General goal}
		\begin{itemize}
			\item Recommend complex objects
			\item Recommend \emph{and} argue
		\end{itemize}
	\end{block}
	\begin{block}{Complex objects}
		\begin{itemize}
			\item Voting rule
			\item Planning
			\item Strategy (game, negociation, …)
			\item Travel itinerary
		\end{itemize}
	\end{block}
	Multi-level argumentation:
	\begin{itemize}
		\item NOT persuasion
		\item NOT predicting the natural user choice
	\end{itemize}
\end{frame}

\subsection{Conclusion}
\begin{frame}
	\frametitle{Conclusion}
	
	\begin{itemize}
		\item A language to express desirable properties of voting rules.
		\item We can then instanciate concrete arguments (example-based).
		\item May render some arguments in the specialized literature accessible to non experts.
		\item Extensions may permit to \emph{debate} about voting rules.
		\item Provides a way to study appreciation of arguments.
	\end{itemize}
\end{frame}

\begin{frame}[plain]
	\addtocounter{framenumber}{-1}
	\begin{center}
		\huge
		\textit{Thank you for your attention!}
	\end{center}
\end{frame}

\appendix
\AtBeginSection{
}

\section{Language}
\begin{frame}
	\frametitle{Example of axiom}
	
	\begin{itemize}
		\item Dominance: if $a$ dominates $b$ in $\prof$, then $b$ may not win.
		\item We want a language to express this kind of axioms.
	\end{itemize}
\end{frame}

\subsection{Presentation}
\begin{frame}
	\frametitle{L-axioms}
	
	\begin{itemize}
		\item Now: “translate” axioms into language-axioms.
		\item An \emph{l-axiom} is a set of formulæ.
	\end{itemize}
\end{frame}

\begin{frame}
	\frametitle{Shortcut notations}
	
	$\powersetz{\allalts}$ the set of subsets of $\allalts$, excluding the empty set.
	\\[2 em]
	Let $\alpha \subseteq \powersetz{\allalts}$ be a set of possible winning alternatives.
	\begin{block}{Uni-profile clause}
		$\latoin$ shortcut for:
		\begin{equation}
			\bigvee_{A \in \alpha} \lato.
		\end{equation}
		\begin{itemize}
			\item Intuitive content.
			\item Called a uni-profile clause.
		\end{itemize}
	\end{block}
\end{frame}

\subsection{L-axioms}
\begin{frame}
	\frametitle{Domain knowledge}
	
	\begin{itemize}
		\item We need some formulæ encoding the voting rule concept.
		\item Define $\kappa$ as the set of all those formulæ.
	\end{itemize}
	\begin{block}{Domain knowledge $\kappa$}
		\setlength\abovedisplayskip{1 ex}
		\begin{enumerate}
			\item a voting rule can’t select more than one set of winners:\\
			for all $\prof$ and all $\emptyset \subset A ≠ B \subseteq \allalts$,
			\begin{equation}
				\lato[\prof][A] ∧ \lato[\prof][B] → \bot.
			\end{equation}
			\item a voting rule must select at least one set of winners:\\
			for all $\prof$,
			\begin{equation}
				\latoin[\prof][\powersetz{\allalts}].
			\end{equation}
		\end{enumerate}
	\end{block}
\end{frame}

\begin{frame}[fragile]
	\frametitle{Fishburn-against-Condorcet argument}
	
	\begin{minipage}{5.5cm}
		\citet[p. 544]{fishburn_paradoxes_1974} argument against the Condorcet principle (see also \url{http://rangevoting.org/FishburnAntiC.html}).
		\begin{block}{Condorcet winner}
%		\setlength\abovedisplayskip{0 ex}
%		\begin{equation}
			$w \text{ VS } \mu, \mu \in \set{a, \ldots, h}\text{? }\uncover<2->{51/101}$
%		\end{equation}
		\end{block}
	\end{minipage}%
	\begin{minipage}{\columnwidth-5cm}
		\small
		\begin{equation}
			\begin{array}{lrrrrrr}
				&\multicolumn{6}{c}{\text{nb voters}}\\
			\cmidrule{2-7}
					&31	&19	&10	&10	&10	&21	\\
			\midrule
				1	&a	&a	&f	&g	&h	&h	\\
				2	&b	&b	&w	&w	&w	&g	\\
				3	&c	&c	&a	&a	&a	&f	\\
				4	&d	&d	&h	&h	&f	&w	\\
				5	&e	&e	&g	&f	&g	&a	\\
				6	&w	&f	&e	&e	&e	&e	\\
				7	&g	&g	&d	&d	&d	&d	\\
				8	&h	&h	&c	&c	&c	&c	\\
				9	&f	&w	&b	&b	&b	&b	\\
			\end{array}
		\end{equation}
	\end{minipage}
	\vspace{-1pt}
	\onslide<3>
	\begin{equation}
		\vspace*{-8.4pt}
		\begin{array}{lrrrrrrrrr}
			&\multicolumn{9}{c}{\text{ranks}}\\
		\cmidrule{2-10}
			&1	&≤2	&≤3	&≤4	&≤5	&≤6	&≤7	&≤8	&≤9	\\
		\midrule
		w	&0	& 30	& 30	& 51	& 51	& 82	& 82	& 82	&101	\\
		a	&50	& 50	& 80	& 80	& 101	& 101	& 101	&101	&101	\\
		\end{array}
	\end{equation}
\end{frame}

\begin{frame}[fragile]
	\frametitle{Fishburn-versus-Condorcet l-axiom}
	
	Define $\prof_F$ the profile shown in the previous slide.
	\begin{definition}[Fishburn-versus-Condorcet]
		The Fishburn-versus-Condorcet l-axiom \laxiom{FvsC} is defined as:
		\setlength\abovedisplayskip{1 ex}
		\begin{equation}
			\latoin[\prof_F][\powersetz{\allalts \setminus \set{w}}].
		\end{equation}
	\end{definition}
\end{frame}

\begin{frame}
	\frametitle{L-axiomatization}
	
	An l-axiomatization is a set of l-axioms.
	
	\begin{definition}[Conforming to $J$]
		The rule $f$ conforms to the l-axiomatization $J$ iff %$f$ conforms to all $j$ in $J$, iff $v_f$ satisfies all $j$, iff 
		$v_f$ assigns the value $\ltru$ to all formulæ in $j$, for all $j \in J$.
	\end{definition}
	An l-axiomatization is consistent iff there exists a voting rule conformant to it.
\end{frame}

\subsection{Arguments}
\begin{frame}
	\frametitle{Arguments}
	
	\begin{definition}[Argument]
		An argument grounded on $J$ is a pair $(\text{\emph{claim}}, \text{\emph{proof}})$,
		\begin{itemize}
			\item $J$ an l-axiomatization,
			\item \emph{claim} a uni-profile clause (thus of the form $\latoin[\prof][\alpha]$),
			\item \emph{proof} a natural deduction proof of the claim grounded on $J$.
		\end{itemize}
	\end{definition}
	\begin{itemize}
		\item The argument shows that for all voting rules $f$ conformant to $J$, $f(\prof)$ selects a set of winners among $\alpha$.
		\item The argument claims that it is only reasonable to choose the winners among $\alpha$ for $\prof$ (provided $J$ is accepted).
		\item \emph{Consistent} arguments require a consistent l-axiomatization.
	\end{itemize}
\end{frame}

\begin{frame}
	\frametitle{Example proof}
	$
	\prof_D =
	\begin{array}{cc}
		a&b\\
		b&c\\
		c&a\\
	\end{array}, 
	\prof_S =
	\begin{array}{cc}
		a&c\\
		b&b\\
		c&a\\
	\end{array}, 
	\prof = \prof_D + \prof_S = 
					\begin{array}{cccc}
						a&b&a&c\\
						b&c&b&b\\
						c&a&c&a\\
					\end{array}
	$.
\begin{enumerate}
	\item $\latoin[\prof_D][\set{\set{a}, \set{b}, \set{a, b}}]$ (\laxiom{Dom})
	\item $\lato[\prof_S][\set{a, b, c}]$ (\laxiom{Sym})
	\item $(\lato[\prof_D][\set{a}] ∧ \lato[\prof_S][\set{a, b, c}]) → \lato[\prof][\set{a}]$ (\laxiom{Reinf})
	\item $(\lato[\prof_D][\set{b}] ∧ \lato[\prof_S][\set{a, b, c}]) → \lato[\prof][\set{b}]$ (\laxiom{Reinf})
	\item $(\lato[\prof_D][\set{a, b}] ∧ \lato[\prof_S][\set{a, b, c}]) → \lato[\prof][\set{a, b}]$ (\laxiom{Reinf})
	\item $\lato[\prof_D][\set{a}] → \lato[\prof][\set{a}]$ (PR from 2 \& 3)
	\item $\lato[\prof_D][\set{b}] → \lato[\prof][\set{b}]$ (PR from 2 \& 4)
	\item $\lato[\prof_D][\set{a, b}] → \lato[\prof][\set{a, b}]$ (PR from 2 \& 5)
	\item $\lato[\prof_D][\set{a}] ∨ \lato[\prof_D][\set{b}] ∨ \lato[\prof_D][\set{a, b}]$ (rewrite 1)
	\item $\lato[\prof][\set{a}] ∨ \lato[\prof][\set{b}] ∨ \lato[\prof][\set{a, b}]$ (PR from 6--9)
	\item $\latoin[\prof][\set{\set{a}, \set{b}, \set{a, b}}]$ (rewrite 10)\qedhere
\end{enumerate}
\end{frame}

\begin{frame}
	\frametitle{Example shortened}
Tweak l-axioms to skip steps which will seem intuitive to humans.
\begin{definition}[Reinforcement-sets]
For each $\prof_1$, $\prof_2$, $\alpha_1, \alpha_2 \subseteq \powersetz{\allalts}$, {\tiny $\cap Q ≠ \emptyset, Q \in \alpha_1 × \alpha_2$}: $(\latoin[\prof_1][\alpha_1] ∧ \latoin[\prof_2][\alpha_2]) → \latoin[\prof_1 + \prof_2][\bigcup_{A_1 \in \alpha_1, A_2 \in \alpha_2} \set{A_1 \cap A_2}]$.
\end{definition}
	\begin{enumerate}
%TODO make first letter of laxioms a capital.
		\item $\latoin[\prof_D][\set{\set{a}, \set{b}, \set{a, b}}]$ (\laxiom{dom})
			\item $\lato[\prof_S][\set{a, b, c}]$ (\laxiom{Sym})
			\item $((1) ∧ (2)) → \latoin[\prof][\set{\set{a}, \set{b}, \set{a, b}}]$ (\laxiom{Reinf-sets})
			\item $\latoin[\prof][\set{\set{a}, \set{b}, \set{a, b}}]$
	\end{enumerate}
\end{frame}

\begin{frame}
	\frametitle{Soundness and completeness}
	
	Consider an l-axiomatization $J$ and a claim $c = \latoin[\prof][\alpha]$.
	\begin{theorem}[Soundness]
		If there exists an argument $\left(c, \text{proof}\right)$ grounded on $J$, the claim holds given $J$.
	\end{theorem}
	\begin{theorem}[Completeness]
		If the claim holds given $J$, then there exists an argument $(c, \text{proof})$ grounded on $J$.
	\end{theorem}
	This is easily obtained from the soundness and completeness of natural deduction in propositional logic.
\end{frame}

\begin{frame}[allowframebreaks]
	\frametitle{Bibliography}
	\def\newblock{\hskip .11em plus .33em minus .07em}
 	\bibliography{zotero}
\end{frame}

\end{document}
